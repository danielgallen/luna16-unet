% !TEX root = ../Report.tex

\subsection{Setup}

Python 3.x has been chosen to design this solution. NiBabel package was used to deal with specificity of medical images. TensorFlow and Keras are additional libraries required to build training models. 

\subsection{Evaluation on training model}
graph loss on training evaluation (dice loss, bce dice loss, batch size (correlates to gpu), epochs, steps per epoch, optimizer)
prediction
image on label and prediction 
evaluating and examining models with hausdorff and mean distance on pictures (graphs or statistics, table)
comparison of unet and deepmedic

explain high hausdorff and how to reduce it

\begin{table}[h!]
	\centering
	\setlength{\tabcolsep}{10pt}
	\renewcommand{\arraystretch}{1.5}
	\begin{tabular}{c c c c}
		\hline 
		Architecture & Dice & Hausdorff & Mean \\
		& Coefficient & distance (mm) & Distance (mm) \\ 
		\hline 
		DeepMedic & 0.968 & 119.50 & 1.45 \\ 
		U-Net & 0.976 & 103.21 & 0.33 \\ 
		\hline
		\newline 
	\end{tabular}
	\caption{Errors of the LSTM network compared to the forecasting approaches of assignment 1.}
	\label{table_result}
\end{table}


\begin{figure}[h!]
	\includegraphics[width=0.49\textwidth, angle=0]{files/jpgunettrain.png}
	\caption{xxx}
	\label{scan_picture}
\end{figure}

\begin{figure}[h!]
	\includegraphics[width=0.49\textwidth, angle=0]{files/deepmedictrain.png}
	\caption{xxx}
	\label{scan_picture}
\end{figure}

\begin{figure}[h!]
	\includegraphics[width=0.49\textwidth, angle=0]{files/preddeepmedic.png}
	\caption{xxx}
	\label{scan_picture}
\end{figure}

\begin{figure}[h!]
	\includegraphics[width=0.49\textwidth, angle=0]{files/predunet.png}
	\caption{xxx}
	\label{scan_picture}
\end{figure}