% !TEX root = ../Report.tex

Since hardware specifications and computational power increased dramatically in the last years machine learning approaches can be applied in various disciplines today. On top of that the progress in research on convolutional neuronal networks (CNN) made it a very powerful tool for image processing where information is gained from image data.\newline
One challenging application is medical image computing (MIC). The main goal of MIC is to extract clinically relevant information or knowledge from medical images. Furthermore Segmentation is the process of partitioning an image into different meaningful segments (e. g. organs, bones, ...).\newline
In this project the goal was to segment the lung of a human body from computed tomography (CT) scans of the LUNA16 dataset \cite{luna}. Every CT scan consist of a variable number of grey-scale image slices. Given these slices a 3D model of the lung needs to be created. To reach this goal every slice of the CT scan will be evaluated separately by a CNN architecture and a label map for the lung and the bronchus on these slices will be predicted. From these outputs a 3D model of the structure can be created. \newline

\begin{figure}[h!]
	\includegraphics[width=0.49\textwidth, angle=0]{files/ctscans.jpg}
	\caption{CT scan from the LUNA16 dataset}
	\label{scan_picture}
\end{figure}

The segmentation of the lung from the rest of the picture is the first step for further image processing. In the LUNA16 dataset the final goal for example is to detect nodules of the lung indicating cancer. Machine learning approaches can be a powerful support for the doctors who treat patients with suspected cancer. The algorithms can reduce human errors, and it might even have the potential to outperform human capabilities or could automatize the process of cancer detection. This could have a positive effect on health care quality and costs.\newline
In this work two different neuronal networks will be trained and tested to segment the lung on the LUNA16 dataset. Furthermore they will be examined and compared under different metrics.\newline
First a short overview on the DeepMedic and the U-Net architecture will be given and will then be related to other approaches for medical image segmentation. After that the process and implementation will be explained and then the results of the two algorithms will be examined and compared. In the end a conclusion on the work will be drawn and further steps sketched.
