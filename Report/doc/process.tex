% !TEX root = ../Report.tex

In this section an detailed description of the dataset will be given. The process of data preprocessing and the training and prediction for the two architectures will be explained.

\subsection{LUNA16 dataset}

The used dataset is from the LUNA16 challenge and each scan contains a number of slices. The algorithm creates a label map for each slice marking every pixel that is part of the lung or the bronchus with a label. An example of one scan and the corresponding labeling is shown in figure \ref{scan_picture}.

\begin{figure}[h!]
	\includegraphics[width=0.49\textwidth, angle=0]{files/Fulllayoutprediction.png}
	\caption{CT scan of the lung and labeled lung (green) and the bronchus (yellow). The first picture represents one slice of the image, the second a 3D representation of the label and the lower two pictures are visualizations of the side and front view.}
	\label{scan_picture}
\end{figure}

The dataset consists of about 10 GB of mdh and raw image files. The files contain a varying number of slices of 512 x 512 pixel grey-scale images and corresponding 512 x 512 pixel label maps. Since computational resources were limited just a subset of the data was used for this work, more specifically 10 CT scans for training and 20 CT scans for evaluation were chosen randomly. Furthermore the more detailed label map of the LUNA16 dataset was reduced to one label for the lungs and one label for the bronchus since the two network architectures are not suitable for representing positional information.\newline

\subsection{Preprocessing of image data}
Each file was converted to a NIfTI image file for better processing and visualization. These files contain all grey-scale image slices of one scan. For more appropriate input values, each image was normalized separately following the standard normal distribution ($\mu = 0.0$ and $\sigma = 1.0$). Every CT scan had to be checked manually since labels in some scans were not properly representing the lungs and the bronchus.\newline
Then, it was necessary to implement data importation to properly use them. Data from directory was converted into input and label matrices following the size $512$ x $512$ x $n$ with the number of all slices in the dataset $n$.

\subsection{Training and validation}
Both network architectures were trained on the 2145 slices of the 10 training CT scans over 35 epochs.\newline
For the DeepMedic architecture a open-source implementation \cite{deepmedic} was modified for the purpose of lung segmentation and the parameters were set up according to successful implementations for brain tumor detection. RMS propagation with a initial learning rate of 0.001, the Nesterov momentum (momentum rate of 0.06) and L2 regularization (regularization rate of 0.0001) were used as an optimizer. For the training a batch size of 10 samples was chosen and the network was evaluated with the binary cross entropy (BCE) as a loss function.\newline
The U-Net architecture was built according to the structure in figure \ref{unetstructure}. For the training process the Adam optimizer with the described loss function $Loss = dice loss + BCE$ from chapter \ref{metrics_chapter} was used in combination with a batch size of 3 slices. To get an idea of the performance of the model during training 10 \% of the 2145 slices were used for validation while the other 90 \% were used for training.\newline

\subsection{Dealing with the big dataset}
Although the dataset was reduced to 10 CT scans some adjustment in the training process still had to be made to handle the size of the dataset.\newline
Therefore, a generator was defined to reduce the data size that will be fed to the training process at once. The generator feeds just a batch of images in real time to the training process instead of loading the whole dataset at once.

\subsection{Prediction, displaying and calculating metrics}

After training the two best models models were used to predict slices of the 20 test CT scans. The predicted labels were again merged to NIfTI files and displayed. For evaluation an average of the dice loss, the Hausdorff distance and the mean distance were calculated over all 20 test scans.